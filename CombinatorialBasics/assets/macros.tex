%%% Tento soubor obsahuje definice různých užitečných maker a~prostředí %%%
%%% Další makra připisujte sem, ať nepřekáží v~ostatních souborech.     %%%

%%% Drobné úpravy stylu

% Tato makra přesvědčují mírně ošklivým trikem LaTeX, aby hlavičky kapitol
% sázel příčetněji a~nevynechával nad nimi spoustu místa. Směle ignorujte.
%\makeatletter
%\def\@makechapterhead#1{
%  {\parindent \z@ \raggedright \normalfont
%   \Huge\bfseries \thechapter. #1
%   \par\nobreak
%   \vskip 20\p@
%}}
%\def\@makeschapterhead#1{
%  {\parindent \z@ \raggedright \normalfont
%   \Huge\bfseries #1
%   \par\nobreak
%   \vskip 20\p@
%}}
%\makeatother

\setlength{\parskip}{0.3em}
\setlist{topsep=0.1em, itemsep=0em}

% Toto makro definuje kapitolu, která není očíslovaná, ale je uvedena v~obsahu.
\def\chapwithtoc#1{
\chapter*{#1}
\addcontentsline{toc}{chapter}{#1}
}

% Trochu volnější nastavení dělení slov, než je default.
\lefthyphenmin=2
\righthyphenmin=2

% Zapne černé "slimáky" na koncích řádků, které přetekly, abychom si
% jich lépe všimli.
\overfullrule=1mm

\renewenvironment{proof}[1][]{
  \par\medskip\noindent
  \textit{\ifthenelse{\equal{#1}{}}
  {Důkaz}
  {#1}}.
}{
\hspace*{\fill}$\qedsymbol$\par\medskip
}

\def\normalipe{0.8}