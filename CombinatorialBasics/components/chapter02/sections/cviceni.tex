\section{Cvičení}

\begin{exercise}\label{exercise:ch02_1}
    \textit{Jana má pět různě barevných triček a tři nestejné sukně. Kolika způsoby si může vzít tričko a sukni, aby pokaždé vypadala jinak?} \citep[str. 145]{Petakova2020}
\end{exercise}
\begin{exercise}\label{exercise:ch02_2}
    V jedné třídě, ve které každý žák ovládá aspoň jeden ze dvou jazyků (angličtinu nebo němčinu), hovoří 25 žáků anglicky, 16 žáků německy a 7 žáků hovoří oběma jazyky. Kolik žáků chodí do této třídy? \citep[sekce Kombinatorika]{kdm2022}
\end{exercise}
\begin{exercise}\label{exercise:ch02_3}
    Určete počet všech přirozených čísel větších než 2000, v jejichž zápisech se vyskytují cifry 1, 2, 4, 6, 8, a to každá nejvýše jednou? \citep[str. 146]{Petakova2020}
\end{exercise}
\begin{exercise}\label{exercise:ch02_4}
    Na běžecké trati běží 8 závodníků. Za předpokladu, že každou z medailí získá právě jeden závodník, vypočítejte, kolik je možností na rozdělení zlaté, stříbrné a bronzové medaile mezi závodníky. \citep[str. 146]{Petakova2020}
\end{exercise}
\begin{exercise}\label{exercise:ch02_5}
    Ve třídě je 30 míst, ale ve třídě 3. B je pouze 28 žáků. Kolika způsoby lze sestavit zasedací pořádek? \citep[str. 146]{Petakova2020}
\end{exercise}
\begin{exercise}\label{exercise:ch02_6}
    Z kolika prvků lze vytvořit 992 variací druhé třídy bez opakování? \citep[str. 146]{Petakova2020}
\end{exercise}
\begin{exercise}\label{exercise:ch02_7}
    Kolika způsoby lze rozmíchat hru 32 karet? \citep[str. 146]{Petakova2020}
\end{exercise}
\begin{exercise}\label{exercise:ch02_8}
    Kolik různých devíticiferných čísel s různými ciframi lze sestavit z cifer 1 až 9? \citep[str. 146]{Petakova2020}
\end{exercise}
\begin{exercise}\label{exercise:ch02_9}
    Kolik přímek určuje deset různých bodů v rovině, z nichž
    \begin{enumerate}[label=(\alph*)]
        \item žádné tři neleží na jedná přímce,
        \item právě šest leží na jedné přímce?
    \end{enumerate}
    \citep[str. 146]{Petakova2020}
\end{exercise}
\begin{exercise}\label{exercise:ch02_10}
    Určete počet všech úhlopříček v konvexním mnohoúhelníku. \citep[str. 147]{Petakova2020}
\end{exercise}
\begin{exercise}\label{exercise:ch02_11}
    V krabici je 10 výrobků, z nichž jsou právě tři vadné. Kolika způsoby lze vybrat 5 výrobků tak, aby:
    \begin{enumerate}[label=(\alph*)]
        \item žádný nebyl vadný,
        \item právě jeden byl vadný,
        \item nejvýše jeden byl vadný,
        \item alespoň dva byly vadné.
    \end{enumerate}
    \citep[str. 147]{Petakova2020}
\end{exercise}
\begin{exercise}\label{exercise:ch02_12}
    Z kolika prvků lze vytvořit 990 kombinací druhé třídy bez opakování? \citep[str. 147]{Petakova2020}
\end{exercise}
\begin{exercise}\label{exercise:ch02_13}
    Učitel chce vytvořit ze čtyř dívek a čtyř chlapců jeden tříčlenný tým, v němž bude jedna dívka a dva chlapci. Kolika různými způsoby může sestavit tým? \citep[příklad \emph{Tříčlenné 69274}]{hackmath2022}
\end{exercise}