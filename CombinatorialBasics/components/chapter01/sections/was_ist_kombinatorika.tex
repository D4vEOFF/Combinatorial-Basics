\section{Was ist kombinatorika?}

\textbf{Kombinatorika} představuje matematickou disciplínu zabývající se se kolekcemi prvků množin s definovanou vnitřní strukturou. Řekneme-li to méně formálně, studuje, kolika způsoby lze sestavit konfiguraci s jistými vlastnostmi. Zároveň se tak váže k blízkému oboru zvanému \textbf{teorie pravděpodobnosti}. \par
Typickou úlohou (otázkou) kombinatoriky je třeba tato:

\begin{exercise}\label{exercise:intro}
    Na svatbě je $n$ lidí.
    \begin{enumerate}[label=(\alph*)]
        \item Kolika způsoby lze $n$ svatebčanů sestavit do řady?
        \item V kolika případech stojí nevěsta napravo od ženicha?
        \item Kolik je řad, že ženich a nevěsta stojí vedle sebe?
    \end{enumerate}
\end{exercise}

Poměrně pěknou a lehce humornou úlohou spadající do klasické kombinatoriky je např. tzv. \emph{problém šatnářky}.

\begin{exercise}\label{exercise:problem_satnarky}
    \textit{Ctihodní pánové v počtu $n$ přijdou na shromáždění, všichni v kloboucích, a odloží si své klobouky do šatny. Při odchodu šatnářka, možná ten den velmi roztržitá, možná dokonce z mizerného osvětlení osleplá, vydá z pánů náhodně jeden z klobouků. Jaká je pravděpodobnost, že žádný pán nedostane od šatnářky zpět svůj klobouk?} \citep[str.~105]{MatousekNesetril2009}
\end{exercise}

Dalšími zajímavými problémy jsou např. tyto

\begin{exercise}
    \begin{itemize}
        \item \textit{Jaký maximální počet oblastí může vzniknout, jestliže pomocí $n$ přímek rozdělíme rovinu?} \citep[str. 38]{Slavik2022}
        \item \textit{Kolika způsoby lze rozměnit jeden dolar?} \citep[str. 130]{HarrisHirstMossinghoff2010}
        \item \textit{Kolik různých náhrdelníků s dvaceti korálky lze vyrobit z korálků rodonitu, růženínu a lazuritu, pokud lze náhrdelník nosit v jakékoli orientaci?} \citep[str. 130]{HarrisHirstMossinghoff2010}
    \end{itemize}
\end{exercise}

Kromě první zmíněné úlohy jsou ostatní svým řešením již nad rámec tohoto textu, avšak např. zmíněný problém šatnářky\footnote{Konkrétně zde se při řešení využije tzv. \emph{princip inkluze a exkluze}.} \ref{exercise:problem_satnarky} vede posléze k velmi zajímavým výsledkům. Nicméně pro řešení úlohy \ref{exercise:intro} a jí příbuzných si v dalších odstavcích vybudujeme potřebný matematický aparát.\par