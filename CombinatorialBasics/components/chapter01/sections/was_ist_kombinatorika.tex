\section{Was ist kombinatorika?}

\textbf{Kombinatorika} představuje matematickou disciplínu zabývající se se kolekcemi prvků množin s definovanou vnitřní strukturou. Řekneme-li to méně formálně, studuje, kolika způsoby lze sestavit konfiguraci s jistými vlastnostmi. Zároveň se tak váže k blízkému oboru zvanému \textbf{teorie pravděpodobnosti}. \par
Typickou úlohou (otázkou) kombinatoriky je třeba tato:

\begin{exercise}
    Na svatbě je $n$ lidí.
    \begin{enumerate}[label=(\alph*)]
        \item Kolika způsoby lze $n$ svatebčanů sestavit do řady?
        \item V kolika případech stojí nevěsta napravo od ženicha?
        \item Kolik je řad, že ženich a nevěsta stojí vedle sebe?
    \end{enumerate}
\end{exercise}

Pro podobné úlohy v dalších odstavcích vybudujeme potřebný matematický aparát.