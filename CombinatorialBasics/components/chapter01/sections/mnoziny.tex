\section{Množiny}

Množiny pro nás budou klíčovým pojmem, neboť s jejich pomocí budeme formulovat další části výkladu. Proto považuji za nezbytné si zopakovat aspoň některé základní vlastnosti a operace, které s množinami můžeme provádět. Množinou v matematice rozumíme "soubor neuspořádaných prvků". Dvě množiny tak považujeme za stejné (sobě rovné) právě tehdy, když mají stejné prvky. Byť tento popis nepředstavuje zcela formální definici, pro naše potřeby s tímto chápáním vystačíme.\par
Množiny zapisujeme pomocí složených závorek $\{,\}$, přičemž jejich specifikace lze provést dvě způsoby:
\begin{itemize}
    \item výčtem (výpisem) jednotlivých prvků,
    \item společnou vlastností
\end{itemize}

\begin{example}
    Množinu $M$ obsahující prvky $a,\,b,\,c$ lze jako
    \begin{equation*}
        M=\set{a,\,b,\,c}.
    \end{equation*}
\end{example}
V případě většího počtu prvků, avšak s jistou strukturou, můžeme množinu specifikovat buď pomocí "\dots" nebo explicitním vyjádřením specifické vlastnosti.

\begin{example}
    Množinu všech přirozených čísel menších nebo rovny 5 lze zapsat jako
    \begin{equation*}
        S=\set{n\in\N \admid n \leqslant 5}
    \end{equation*}
\end{example}

Důležitou vlastností množin je, že neuvažujeme násobné výskyty prvků. Tedy např. množiny $M=\set{1,\,2,\,3}$ a $N=\set{1,\,2,\,2,\,3,\,3}$ jsou si rovny, tj. $M=N$.