\section{Množiny}

Množiny pro nás budou klíčovým pojmem, neboť s jejich pomocí budeme formulovat další části výkladu. Proto považuji za nezbytné si zopakovat aspoň některé základní vlastnosti a operace, které s množinami můžeme provádět. Množinou v matematice rozumíme "soubor neuspořádaných prvků". Dvě množiny tak považujeme za stejné (sobě rovné) právě tehdy, když mají stejné prvky. Byť tento popis nepředstavuje zcela formální definici, pro naše potřeby s tímto chápáním vystačíme.\par
Množiny zapisujeme pomocí složených závorek $\{,\}$, přičemž jejich specifikace lze provést dvě způsoby:
\begin{itemize}
    \item výčtem (výpisem) jednotlivých prvků,
    \item společnou vlastností
\end{itemize}

\begin{example}
    Množinu $M$ obsahující prvky $a,\,b,\,c$ lze jako
    \begin{equation*}
        M=\set{a,\,b,\,c}.
    \end{equation*}
\end{example}
V případě většího počtu prvků, avšak s jistou strukturou, můžeme množinu specifikovat buď pomocí "\dots" nebo explicitním vyjádřením specifické vlastnosti.

\begin{example}
    Množinu všech přirozených čísel menších nebo rovny 5 lze zapsat jako
    \begin{equation*}
        S=\set{n\in\N \admid n \leqslant 5}\;\;\;\text{nebo}\;\;\;S=\set{1,\,2,\,\dots,\,5}.
    \end{equation*}
\end{example}

Důležitou vlastností množin je, že neuvažujeme násobné výskyty prvků. Tedy např. množiny $M=\set{1,\,2,\,3}$ a $N=\set{1,\,2,\,2,\,3,\,3}$ jsou si rovny, tj. $M=N$. Též je vhodné si připomenout, že chceme-li vyjádřit, že libovolný prvek $a$ je v množině $A$, pak píšeme $a \in A$ (čteme "$a$ náleží množině $A$"). Naopak v případě, že prvek $a$ nenáleží množině $A$, píšeme $a \notin A$.\par
Podstatnou vlastností pro nás budou operace \emph{sjednocení}, \emph{průniku} a \emph{rozdílu} množin.

\begin{definition}[Sjenocení, průnik a rozdíl]\label{def:mnozinove_operace}
    Mějme množiny\footnote{Mohou být \textbf{konečné} i \textbf{nekonečné}, avšak nás budou zajímat konečné množiny.} $A,\,B$. Pak definujeme:
    \begin{enumerate}[label=(\roman*)]
        \item sjednocení $A \cup B = \set{x \admid x\in A \lor x\in B}$, tj. výsledná množina obsahuje prvky množiny $A$ a zároveň prvky množiny $B$.
        \item průnik $A \cap B = \set{x \admid x\in A \land x\in B}$, tj. výsledná množina obsahuje \emph{pouze} prvky, které náleží oběma množinám.
        \item rozdíl $A \setminus B = \set{x \admid x\in A \land x \notin B}$, tj. výsledná množina obsahuje pouze ty prvky z množiny $A$, které nenáleží množině $B$.
    \end{enumerate}
\end{definition}

\begin{example}
    Pro množiny\footnote{U množiny $Y$ si uvědomme, že prvek $\set{z}$ není to samé jako prvek $z$, tedy např. po sjednocení se ve výsledné množině vyskytnou oba.} $X=\set{x,\,y,\,z}$ a $Y=\set{x,z,\set{z},w}$ platí
    \begin{itemize}
        \item $X \cup Y = \set{x,\,y,\,z} \cup \set{x,z,\set{z},w}=\set{x,\,y,\,z,\,x,z,\set{z},w}=\set{x,\,y,\,z,\,\set{z},w}$,
        \item $X \cap Y = \set{x,\,y,\,z} \cap \set{x,z,\set{z},w}=\set{x,\,z}$,
        \item $X \setminus Y = \set{x,\,y,\,z} \setminus \set{x,z,\set{z},w}=\set{x,\,z}=\set{y}$.
    \end{itemize}
    Zkuste si výsledky operací porovnat s definicí \ref{def:mnozinove_operace} výše.
\end{example}

Pro větší počet množin můžeme využít pro zápis sjednocení tzv. \emph{velké operátory} $\bigcup,\,\bigcap$. Máme-li tedy množiny $X_1,\,X_2,\,\dots,\,X_n$, můžeme jejich sjednocení, resp. průnik zapsat jako
\begin{equation*}
    \bigcup\limits_{i=1}^{n}X_i = X_1 \cup X_2 \cup \dots \cup X_n\;\;\;\text{resp.}\;\;\;\bigcap\limits_{i=1}^{n}X_i = X_1 \cap X_2 \cap \dots \cap X_n
\end{equation*}

Poslední, co nás bude zajímat, je velkost množiny. Tu budeme označovat svislými čarami, tedy např. velikost množiny $X$ zapíšeme jako $\sizeof{X}$. Konkrétně např. pro množinu $A=\set{-1,\,0,10,20}$ je velikost $\sizeof{A}=4$.