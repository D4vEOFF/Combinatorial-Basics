\section{Cvičení}
\begin{exercise}\label{exercise:ch04_1}
    \textit{V autobuse cestuje 32 cestujících, z toho tři pasažéři bez platné jízdenky. Do autobusu nastoupil revizor a po chvíli začal kontrolovat lístky. Jaká je pravděpodobnost, že jako prvnímu chtěl zkontrolovat jízdenku právě pasažérovi bez lístku?} \cite{hackmath2022}
\end{exercise}
\begin{exercise}\label{exercise:ch04_2}
    \textit{Hodíme třikrát kostkou. Vypočítejte pravděpodobnost, že při prvním hodu padne sudé číslo, při druhém hodu padne liché číslo a při třetím hodu padne šestka.} \citep[str. 171]{Petakova2020}
\end{exercise}
\begin{exercise}\label{exercise:ch04_3}
    \textit{Ve třídě je 30 žáků. Právě pět z nich nemá domácí úkol. Učitel náhodně kontroluje 6 žáků. Vypočítejte pravděpodobnost, že nejvýše dva žáci, které učitel kontroluje, nemají domácí úkol.} \citep[str. 171]{Petakova2020}
\end{exercise}
\begin{exercise}\label{exercise:ch04_4}
    \textit{Žárovka svítí se spolehlivostí $92\,\%$. Jaká je pravděpodobnost zařízení, ve kterém jsou tři žárovky zapojeny sériově.} \citep[str. 172]{Petakova2020}
\end{exercise}
\begin{exercise}\label{exercise:ch04_5}
    \textit{V osudí je 5 červených a 3 bílé koule. Z osudí v prvním tahu vytáhneme jednu kouli, při druhém tahu vytáhneme opět jednu kouli. S jakou pravděpodobností vytáhneme ve druhém tahu červenou kouli, jestliže po prvním tahu kouli}
    \begin{enumerate}[label=(\alph*)]
        \item \textit{vracíme,}
        \item \textit{nevracíme?}
    \end{enumerate}
    \citep[str. 173]{Petakova2020}
\end{exercise}
\begin{exercise}\label{exercise:ch04_6}
    \textit{Loterie má 10 000 losů, z nichž právě 20 vyhrává. S jakou pravděpodobností alespoň něco vyhrajeme, koupíme-li si 4 losy?} \citep[str. 174]{Petakova2020}
\end{exercise}