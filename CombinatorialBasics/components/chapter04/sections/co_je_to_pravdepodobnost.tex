\section{Co je to pravděpodobnost?}

\textbf{Pravděpodobnost} (někdy též \emph{šance}) je jeden z pojmů, který vystupuje nejen v matematice, ale i v reálném životě. Když si hodíme mincí, asi je nám jasné, že je stejná šance, že nám padne rub nebo líc. Naopak třeba u hodu klasicky hrací kostkou máme šanci $1:6$, že nám padne jednička a šance tak již není vyrovnaná situaci, že nám nepadne. Intuitivně tak asi tušíme, co se myslí, když se řekne, že např. šance výhry je 75 \%. Co však ale reprezentuje ono číslo? Definovat, co je to "skutečná" pravděpodobnost je obtížné a jedná se spíše o záležitost filozofickou. V matematice se však tomuto problému dokážeme vyhnout.\par
Jako lidé chápeme, že šance výhry 70 \% je v jistém smyslu "lepší" než např. 30 \%, ale proč tomu tak je? Pro příklad si vezměme již zmíněnou klasickou hrací kostku. Pro simulaci hodů využijeme jednoduchý program v jazyce Python:
\VerbatimInput{components/chapter04/sections/dice.py}

Níže je tabulka četností výskytů a jejich poměru vůči celkovému počtu jednotlivých ok po miliónu pokusů.
\begin{table}[H]
    \centering
    \begin{tabular}{|c|c|c|}
    \hline
    Počet ok & Četnost & Výskyt (\%)     \\ \hline
    1     & 166 896 & $16{,}6896$ \\
    2     & 166 789 & $16{,}6789$ \\
    3     & 166 508 & $16{,}6508$ \\
    4     & 166 863 & $16{,}6863$ \\
    5     & 166 151 & $16{,}6150$ \\
    6     & 166 794 & $16{,}6794$ \\ \hline
    \end{tabular}
    \caption {Četnost hodů ok na hrací kostce po miliónu pokusech.}
\end{table}
Víme, že šance pro všechny strany je stejná - $1:6$, což odpovídá hodnotě numerické hodnotě $0{,}1\overline{6}$, neboli $16{,}\overline{6}$. Celkově tak můžeme vidět, že procentuální výskyt se jen málo liší od našeho očekávání. Tímto způsobem můžeme vnímat pravděpodobnost. Jako číslo ke kterému se blíží \textbf{poměr počtu příznivých pokusů a počtu všech pokusů}.\par
Praktický výpočet pravděpodobnosti však probíhá trochu jinak. U kostky jsme určili pravděpodobnost hodu jedničky tak, že jsme vzali \textbf{počet příznivých hodů} ku \textbf{počtu všech možných hodů}. A v tomto duchu také definujeme pravděpodobnost.\par
\begin{definition}[Elementární a náhodný jev, pravděpodobnost]\label{def:elementarni_nahodny_jev_pravdepodobnost}
    Mějme množinu $\Omega=\set{\omega_1,\,\omega_2,\,\dots,\,\omega_n}$. Pak
    \begin{itemize}
        \item $\Omega$ nazýváme \textbf{množinou elementárních jevů} a libovolný její prvek $\omega_i$ \textbf{elementárním jevem}.
        \item libovolnou podmnožinu $A\subseteq\Omega$ nazýváme \textbf{(náhodný) jev}.
    \end{itemize}
    Dále definujeme funkci $\Prob$ přiřazující libovolnému jevu $A\subseteq\Omega$ reálné číslo z intervalu $\langle0,\,1\rangle$ splňující
    \begin{enumerate}[label=(\roman*)]
        \item $\Prob(\emptyset)=0$,
        \item $\Prob(\Omega)=1$,
        \item $\Prob(A\cup B)=\Prob(A)+\Prob(B)$,
    \end{enumerate}
    kde $A,\,B\subseteq\Omega$ jsou disjunktní jevy. Číslo $\Prob(A)$ nazýváme \textbf{pravděpodobností jevu $A$}.
\end{definition}
Poslední vztah lze zobecnit pro obecný počet disjunktních jevů $A_1,\,A_2,\,\dots,\,A_n$, tj.
\[\Prob\left(\bigcup\limits_{i=1}^{n}A_i\right)=\sum_{i=1}^{n}\Prob(A_i).\]
Ve všech příkladech dále budeme pracovat s variantou, že všechny elementární jevy jsou stejně pravděpodobné. To má tedy za následek způsob výpočtu, který jsme zmínili výše, tedy je-li $A\subseteq\Omega$ libovolný jev, pak pravděpodobnost jevu $A$ stanovíme jako
\[\Prob(A)=\dfrac{\sizeof{A}}{\sizeof{\Omega}}.\]
\begin{example}
    V případě hodu kostkou tvoří množinu elementárních jevů $\Omega$ případy, kdy 1 až 6 ok. Označíme-li si jednotlivé počty ok $O_1,\,\dots,\,O_6$, pak $\Omega=\set{O_1,\,\dots,\,O_6}$. Všechny elementární jevy jsou zde stejně pravděpodobné, tedy pro libovolné $i\in\set{1,\,2,\,\dots,\,6}$ platí
    \[\Prob(O_i)=\dfrac{1}{6}.\]
    Vezmeme-li jev $L$, který nastane právě tehdy, když padne lichý počet ok, pak $L=\set{O_1,\,O_3,\,O_5}$ (všechny elementární jevy jsou \textbf{vždy} považovány za vzájemně po dvou disjunktní). Pravděpodobnost jevu $L$ je podle bodu \textit{(iii)} v definici \ref{def:elementarni_nahodny_jev_pravdepodobnost} rovna
    \[\Prob(L)=\Prob(O_1)+\Prob(O_3)+\Prob(O_5)=\dfrac{1}{6}+\dfrac{1}{6}+\dfrac{1}{6}=\dfrac{1}{2}=50\,\%,\]
    což souhlasí s naším očekáváním.
\end{example}
\begin{task}
    Z osmnácti lístků označených čísly 1 - 18 vytáhneme náhodně jeden lístek. Jaká je pravděpodobnost, že na vytažením lístku bude:
    \begin{enumerate}[label=(\alph*)]
        \item sudé číslo
        \item číslo dělitelné 3
        \item prvočíslo
        \item dělitelné 6
    \end{enumerate}
    \citep[sekce \emph{Pravděpodobnost a statistika}]{prikladyeu2022}
\end{task}
\begin{solution}
    Zde představuje množinu elementárních jevů, že si vytáhneme jeden lístek z osmnácti, tj. $\Omega=\set{L_1,\,L_2,\,\dots,\,L_{18}}$, kde $L_i$ je jev, kdy jsme si vytáhli lístek s číslem $i$ (všechny tahy jsou stejně pravděpodobné). Vyřešíme po řadě jednotlivé pravděpodobnosti.\par
    \begin{enumerate}[label=(\alph*)]
        \item Jev $A$ bude představovat situaci, kdy jsme si vytáhli sudé číslo, tzn. $A=\set{L_2,\,L_4,\,\dots,\,L_{18}}$. Počet příznivých jevů je tak
        \[\sizeof{A}=\dfrac{18}{2}=9.\]
        Počet všech možných jevů je $\sizeof{\Omega}=18$. Tedy
        \[\Prob(A)=\dfrac{\sizeof{A}}{\sizeof{\Omega}}=\dfrac{9}{18}=\dfrac{1}{2}=50\,\%.\]
        \item Počet čísel dělitelných třemi z množiny $\set{1,\,2,\,\dots,\,18}$ je $18/3=6$, tedy označíme-li si tento jev $B$, pak $\sizeof{B}=6$. Tzn.
        \[\Prob(B)=\dfrac{\sizeof{B}}{\sizeof{\Omega}}=\dfrac{6}{18}=\dfrac{1}{3}=33{,}\overline{3}\,\%.\]
        \item Z množiny $\set{1,\,2,\,\dots,\,18}$ jsou prvočísla $2,\,3,\,5,\,7,\,11,\,13$ a $17$, tj. pro příslušný jev $C$ platí $\sizeof{C}=7$. Tzn.
        \[\Prob(C)=\dfrac{\sizeof{C}}{\sizeof{\Omega}}=\dfrac{7}{18}=38{,}\overline{8}\,\%.\]
        \item Počet čísel dělitelných šesti je přesně 3 (jsou jimi 6, 12 a 18), tj. $\sizeof{D}=3$ a tedy
        \[\Prob(D)=\dfrac{\sizeof{D}}{\sizeof{\Omega}}=\dfrac{3}{18}=\dfrac{1}{6}=16{,}\overline{6}\,\%.\]
    \end{enumerate}
\end{solution}
\begin{task}
    Jaká je pravděpodobnost že při hodu dvěma kostkami (červené a modré) padne součet 8? \citep[sekce \emph{Pravděpodobnost a statistika}]{prikladyeu2022}
\end{task}
\begin{solution}
    Zde je třeba se trochu promyslet, co považovat za elementární jev. Pokud hodíme dvěma kostkami, obdržíme dvojici hodnot $(i,\,j)$, kde na modré kostce padlo číslo $i$ a na červené kostce číslo $j$. Tedy množina elementárních jevů množina všech takových dvojic, tj. $\Omega=\set{(i,\,j) \admid 1 \leqslant i,\,j \leqslant 6}$. Dvojice, u nichž je součet ok na obou kostkách roven osmi, jsou $(2,\,6),\,(3,\,5),\,(4,\,4),\,(5,\,3),\,(6,\,2)$ a tedy $A=\set{(2,\,6),\,(3,\,5),\,(4,\,4),\,(5,\,3),\,(6,\,2)}$. Tedy takových dvojic je celkem 5. Výsledná pravděpodobnost je
    \[\Prob(A)=\dfrac{\sizeof{A}}{\sizeof{\Omega}}=\dfrac{5}{6^2}=\dfrac{5}{36}=13{,}\overline{8}\,\%.\]
\end{solution}
\begin{remark}
    Někdy nepotřebujeme nutně vypočítat přímo pravděpodobnost, že nastane jev $A$, ale že naopak nenastane. To znamená, že nastane kterýkoliv z elementárních jevů, které nenáleží jevu $A$. Tomu odpovídá doplněk množiny $A$ do množiny $\Omega$, tj. $A^\complement$. Takovému jevu říkáme \textbf{jev opačný k $A$} a platí
    \[\Prob\left(A^\complement\right)=1-\Prob(A).\]
\end{remark}
\begin{task}
    V sáčku je 10 skleněnek a 20 hliněnek. Náhodně vybereme 7 kuliček. Jaká je pravděpodobnost, že ve výběru se vyskytne \textbf{alespoň jedna skleněnka}?
\end{task}
\begin{solution}[Řešení první]
    Vybíráme neuspořádané sedmice z celkově třiceti kuliček. Takových výběrů existuje $\sizeof{\Omega}=\binom{30}{7}$ Zkusíme přímo spočítat všechny možnosti. Nejdříve vybereme skleněnky a počet doplníme hliněnkami do sedmi. Jevy $K_1,\,K_2,\,\dots,\,K_7$, kde $K_i$ jsou výběry s právě $i$ skleněnkami, jsou jistě disjunktní. Tedy stačí sečíst všechny možnosti takových výběrů skleněnek a následně výběr doplnit všemi $(7-i)$-ticemi hliněnek. Tedy
    \[\sizeof{\bigcup\limits_{i=1}^{7}K_i}=\sum_{i=1}^{7}\binom{10}{i}\binom{20}{7-i}\]
    Zaměříme-li se na sumu na pravé straně, lze si všimnout, že můžeme aplikovat Vandermondovu identitu \ref{thm:vandermondova_identita}, kde $n=10$, $m=20$ a $r=7$. Musíme si však dát pozor, neboť v původním znění věty bereme index $k$ od nuly. Tedy zde nám chybí člen $\binom{10}{0}\binom{20}{7}$ (pro $i=0$).
    \begin{align*}
        \sum_{i=1}^{7}\binom{10}{i}\binom{20}{7-i}&=\sum_{i=1}^{7}\binom{10}{i}\binom{20}{7-i}+\binom{10}{0}\binom{20}{7}-\binom{10}{0}\binom{20}{7}\\
        &\stackrel{\ref{thm:vandermondova_identita}}{=}\binom{10+20}{7}-\binom{10}{0}\binom{20}{7}=\binom{30}{7}-\binom{20}{7}.
    \end{align*}
    Tedy výsledná pravděpodobnost je
    \[\Prob\left(\bigcup\limits_{i=1}^{7}K_i\right)=\dfrac{1}{\sizeof{\Omega}}\cdot\sizeof{\bigcup\limits_{i=1}^{7}K_i}=\dfrac{\binom{30}{7}-\binom{20}{7}}{\binom{30}{7}}\approx 0{,}962=96{,}2\,\%.\]
\end{solution}
Výpočet je jistě správný, nicméně existuje lehčí způsob, jak dojít ke stejnému výsledku pomocí uvážení opačného jevu.
\begin{solution}[Řešení druhé]
    V původním řešení jsme museli přímo spočítat počet všech možných výběrů postupně s jednou, dvěma, třemi atd. až sedmi skleněnkami. Avšak existuje jednoduší postup. Zkusíme naopak vypočítat pravděpodobnost, že výběr sedmi kuliček nebude obsahovat žádnou skleněnku. To odpovídá jednoduše výběru sedmice hliněnek a tedy jich existuje $\binom{20}{7}$. Pokud si tento jev označíme $K_0$, pak jeho pravděpodobnost
    \[\Prob(K_0)=\dfrac{\sizeof{K_0}}{\sizeof{\Omega}}=\dfrac{\binom{20}{7}}{\binom{30}{7}}\approx 0{,}038=3{,}8\,\%.\]
    Protože $K_0$ je opačný k jevu $\bigcup_{i=1}^{7}K_i$, pak
    \[\Prob\left(\bigcup\limits_{i=1}^{7}K_i\right)=1-\Prob(K_0)=1-0{,}038=0{,}962=96{,}2\,\%.\]
\end{solution}
\begin{task}
    Heslo se skládá z 10 symbolů latinské abecedy, tzn.
    \begin{itemize}
        \item 26 velkých písmen,
        \item 26 malých písmen a
        \item 10 číslic,
    \end{itemize}
    přičemž žádná z číslic se neopakuje. Jaká je pravděpodobnost, že při náhodném vygenerování bude heslo obsahovat alespoň jednu číslici? (Úloha s řešením převzata z \cite{Tiser2022}.)
\end{task}
\begin{solution}
    Obdobně i zde bychom tuto úlohu mohli řešit dvěma způsoby, nicméně pokud porovnáme vstupní hodnoty s předešlou úlohou, nejspíše bychom došli k závěru, že varianta řešení pomocí přímého spočítání jednotlivých hesel, by zde byla o dost náročnější. Proto zvolíme opět opačný postup, tedy spočítáme všechna hesla, která neobsahují \textbf{žádnou číslici}. Jev, kdy bylo vygenerováno heslo s alespoň jednou číslicí, si označíme $N$, tzn. hledáme $N^\complement$. Takových hesel existuje $\sizeof{N^\complement}=\binom{52}{10}$. Celkový počet možný hesel je $\sizeof{\Omega}=\binom{62}{10}$.\par
    Pravděpodobnost jevu opačnému k $N$ je tak
    \[\Prob\left(N^\complement\right)=\dfrac{\sizeof{N}}{\sizeof{\Omega}}=\dfrac{\binom{52}{10}}{\binom{62}{10}}\approx 0{,}147=14{,}7\,\%\]
    a hledaná pravděpodobnost jevu $N$ je tedy
    \[\Prob(N)=1-\Prob\left(N^\complement\right)=1-0{,}147=0{,}853=85{,}3\,\%.\]
\end{solution}