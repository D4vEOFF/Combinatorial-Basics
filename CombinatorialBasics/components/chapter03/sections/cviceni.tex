\section{Cvičení}

\begin{exercise}\label{exercise:ch03_1}
    Vyjádřete výraz
    \[\binom{10}{1}+\binom{10}{0}+\binom{11}{9}\]
    jedním kombinačním číslem.
\end{exercise}
\begin{exercise}\label{exercise:ch03_2}
    Řešte rovnici
    \[\binom{n}{n-2}+\binom{n}{n-1}=\dfrac{n^2+1}{2}.\]
\end{exercise}
\begin{exercise}\label{exercise:ch03_3}
    Řešte rovnici
    \[\binom{n}{3}+\binom{n}{2}=15(n-1).\]
\end{exercise}
\begin{exercise}\label{exercise:ch03_4}
    Vypočítejte pátý člen binomického rozvoje $(1+y)^{10}$.
\end{exercise}
\begin{exercise}\label{exercise:ch03_5}
    Určete $x\in\R$ tak, aby pátý člen binomického rozvoje $\displaystyle\left(\dfrac{2}{x}-\sqrt{x}\right)$ byl roven 2016.
\end{exercise}
\begin{exercise}\label{exercise:ch03_6}
    Který člen binomického rozvoje $(5-2m)^7$ obsahuje $m^4$?
\end{exercise}